\documentclass[12pt,a4paper]{article}
\usepackage[utf8]{inputenc}
\usepackage[french]{babel}
\usepackage{amsmath}
\usepackage{amsfonts}
\usepackage{amssymb}
\usepackage{graphicx}
\usepackage{xcolor}
\usepackage{hyperref}
\usepackage{tikz}
\usepackage{geometry}
\usepackage{float}
\usepackage{caption}

\geometry{margin=2.5cm}

\hypersetup{
    colorlinks=true,
    linkcolor=blue,
    filecolor=magenta,
    urlcolor=cyan,
}

\title{Calcul des Périodes d'Éclipse Satellite avec Orekit}
\author{SenSat}
\date{\today}

\begin{document}

\maketitle

\tableofcontents
\newpage

\section{Introduction aux Éclipses Satellitaires}

Une \textbf{éclipse satellitaire} se produit lorsqu'un satellite en orbite autour de la Terre entre dans l'ombre ou la pénombre de la Terre. Ce phénomène a d'importantes implications pour les opérations satellitaires, notamment en ce qui concerne la production d'énergie par les panneaux solaires et la gestion thermique du satellite.

\subsection{Types d'Éclipses}

Pour un satellite en orbite terrestre, on distingue deux types d'éclipses:

\begin{itemize}
    \item \textbf{Éclipse d'ombre (Umbra)}: Le satellite est complètement dans l'ombre de la Terre et aucun rayon solaire direct ne l'atteint.
    \item \textbf{Éclipse de pénombre (Penumbra)}: Le satellite est partiellement dans l'ombre de la Terre, recevant une fraction réduite de la lumière solaire.
\end{itemize}

\begin{figure}[H]
    \centering
    \begin{tikzpicture}
        \draw[fill=yellow!20] (0,0) circle (1.5);
        \draw[->] (1.5,0) -- (3,0);
        \draw[->] (1.5,0.5) -- (8,0.5);
        \draw[->] (1.5,-0.5) -- (8,-0.5);
        \draw (0,0) node {Soleil};
        
        \draw[fill=blue!30] (5,0) circle (0.8);
        \draw (5,0) node {Terre};
        
        \draw[fill=black!20] (5,0) -- (8,0.5) -- (8,-0.5) -- cycle;
        \draw[fill=black!10] (5,0) -- (8,1) -- (8,-1) -- cycle;
        
        \draw (6.7,0) node[circle,fill=red,inner sep=1.5pt]{};
        \draw (6.7,0) node[above] {Satellite};
        
        \draw (7,0.3) node {Ombre};
        \draw (7.5,0.8) node {Pénombre};
    \end{tikzpicture}
    \caption{Schéma simplifié des zones d'ombre et de pénombre}
    \label{fig:eclipse_schema}
\end{figure}

\subsection{Importance des Calculs d'Éclipse}

Calculer précisément les périodes d'éclipse est crucial pour:
\begin{itemize}
    \item Gérer l'alimentation du satellite (dimensionnement des batteries)
    \item Planifier les opérations qui nécessitent une puissance importante
    \item Assurer le contrôle thermique du satellite
    \item Prévoir les pertes de communication potentielles
\end{itemize}

\section{Fondements Mathématiques}

\subsection{Modèle Géométrique}

Le modèle géométrique de base pour la détection d'éclipses repose sur la position relative de trois corps célestes: le Soleil (source de lumière), la Terre (corps occultant) et le satellite.

\subsubsection{Conditions Géométriques d'Éclipse}

Pour déterminer si un satellite est en éclipse, nous devons vérifier:

\begin{enumerate}
    \item Si la Terre se trouve entre le Soleil et le satellite
    \item Si la distance angulaire entre le centre de la Terre et le Soleil, vue depuis le satellite, est inférieure à un certain seuil
\end{enumerate}

Mathématiquement, soit:
\begin{itemize}
    \item $\vec{r}_S$: vecteur position du Soleil dans un repère inertiel
    \item $\vec{r}_E$: vecteur position de la Terre dans ce même repère
    \item $\vec{r}_{sat}$: vecteur position du satellite
\end{itemize}

Pour qu'une éclipse soit possible, on doit avoir:
\begin{equation}
(\vec{r}_{sat} - \vec{r}_E) \cdot (\vec{r}_S - \vec{r}_E) < 0
\end{equation}

Cette condition vérifie que le satellite et le Soleil sont de côtés opposés par rapport à la Terre.

\subsection{Calcul de l'Angle d'Occultation}

L'angle d'occultation détermine si le satellite est dans l'ombre, la pénombre ou en pleine lumière:

\begin{equation}
\alpha = \arcsin\left(\frac{R_E}{|\vec{r}_{sat} - \vec{r}_E|}\right) + \arcsin\left(\frac{R_S}{|\vec{r}_S - \vec{r}_{sat}|}\right)
\end{equation}

Où:
\begin{itemize}
    \item $R_E$ est le rayon de la Terre
    \item $R_S$ est le rayon du Soleil
\end{itemize}

Si l'angle entre les directions Terre-Satellite et Terre-Soleil est:
\begin{itemize}
    \item Inférieur à $\alpha$: le satellite est en éclipse totale (ombre)
    \item Compris entre $\alpha$ et $\alpha + \epsilon$: le satellite est en éclipse partielle (pénombre)
    \item Supérieur à $\alpha + \epsilon$: le satellite est en pleine lumière
\end{itemize}

\section{Implémentation dans Orekit}

Orekit est une bibliothèque open-source écrite en Java qui fournit des algorithmes précis pour la mécanique spatiale et l'astrodynamique. Elle inclut des fonctionnalités pour la détection des éclipses.

\subsection{Modèle de Données de Satellite}

Orekit utilise les données TLE (Two-Line Element) pour initialiser l'orbite d'un satellite:

\begin{verbatim}
ISS (ZARYA)
1 25544U 98067A   21086.42859439  .00000318  00000-0  16195-4 0  9991
2 25544  51.6445 354.6522 0003156  48.6176  25.4760 15.48939243277469
\end{verbatim}

Ces données contiennent les éléments orbitaux qui définissent complètement l'orbite du satellite à un instant donné (l'époque).

\subsection{Détection d'Éclipses dans Orekit}

Orekit utilise un système d'événements pour détecter les éclipses. L'algorithme de détection d'éclipses se décompose en plusieurs étapes:

\subsubsection{Initialisation des Modèles}

\begin{enumerate}
    \item Configuration du propagateur orbital pour simuler le mouvement du satellite
    \item Définition des corps célestes impliqués (Terre, Soleil)
    \item Configuration du détecteur d'éclipses avec les paramètres appropriés
\end{enumerate}

\subsubsection{Détection des Événements}

Orekit modélise une éclipse comme un "événement" qui se produit lorsque le satellite entre ou sort de l'ombre. Mathématiquement, un événement est détecté lorsqu'une fonction $g(t)$ change de signe:

\begin{equation}
g(t) = 
\begin{cases}
< 0 & \text{si le satellite est dans l'ombre} \\
> 0 & \text{si le satellite est en pleine lumière}
\end{cases}
\end{equation}

Dans Orekit, cette fonction est implémentée dans la classe \texttt{EclipseDetector}.

\subsection{Exemple de Code Orekit}

Voici comment le détecteur d'éclipses est configuré dans Orekit:

\begin{verbatim}
// Obtention des corps célestes
PVCoordinatesProvider sun = CelestialBodyFactory.getSun();
PVCoordinatesProvider earth = CelestialBodyFactory.getEarth();

// Création du détecteur d'éclipses
double earthRadius = Constants.RIGID_EARTH_SEMI_MAJOR_AXIS;
EclipseDetector eclipseDetector = new EclipseDetector(sun, earthRadius, earth);

// Configuration du gestionnaire d'événements
eclipseDetector.withHandler(new EventHandler<EclipseDetector>() {
    @Override
    public Action eventOccurred(SpacecraftState s, EclipseDetector detector, 
                              boolean increasing) {
        if (increasing) {
            // Le satellite sort de l'éclipse
            eclipseExits.add(s.getDate());
        } else {
            // Le satellite entre en éclipse
            eclipseEntries.add(s.getDate());
        }
        return Action.CONTINUE;
    }
});

// Ajout du détecteur au propagateur
propagator.addEventDetector(eclipseDetector);
\end{verbatim}

\section{Dynamique Orbitale et Calculs d'Éclipses}

\subsection{Propagation d'Orbite}

Pour calculer les éclipses sur une période donnée, Orekit doit propager l'orbite du satellite dans le temps. Cette propagation utilise des équations différentielles qui tiennent compte:

\begin{itemize}
    \item De la gravité terrestre (modèle du géopotentiel)
    \item Des perturbations dues au Soleil et à la Lune (attraction gravitationnelle)
    \item De la pression de radiation solaire
    \item De la traînée atmosphérique
    \item Des autres forces perturbatrices
\end{itemize}

L'équation fondamentale est:

\begin{equation}
\ddot{\vec{r}} = -\mu \frac{\vec{r}}{r^3} + \sum \vec{a}_{\text{perturb}}
\end{equation}

Où:
\begin{itemize}
    \item $\vec{r}$ est le vecteur position du satellite
    \item $\mu$ est le paramètre gravitationnel de la Terre
    \item $\sum \vec{a}_{\text{perturb}}$ représente la somme des accélérations perturbatrices
\end{itemize}

\subsection{Intégration Numérique}

Orekit utilise des intégrateurs numériques avancés pour résoudre ces équations différentielles. Les intégrateurs les plus couramment utilisés sont:

\begin{itemize}
    \item \textbf{Dormand-Prince} (ordre 8): Intégrateur à pas variable de haute précision
    \item \textbf{Runge-Kutta}: Famille d'intégrateurs à pas fixe
    \item \textbf{Adams-Bashforth-Moulton}: Intégrateur multi-pas pour les problèmes non raides
\end{itemize}

\subsection{Calcul des Périodes d'Éclipses}

Le calcul des périodes d'éclipses s'effectue en:

\begin{enumerate}
    \item Propageant l'orbite du satellite sur la période d'intérêt
    \item Détectant les moments d'entrée et de sortie d'éclipse
    \item Calculant la durée de chaque éclipse comme la différence entre les temps de sortie et d'entrée
\end{enumerate}

Soit $t_{\text{entrée},i}$ et $t_{\text{sortie},i}$ les instants d'entrée et de sortie de la $i$-ème éclipse, la durée de cette éclipse est donnée par:

\begin{equation}
\Delta t_i = t_{\text{sortie},i} - t_{\text{entrée},i}
\end{equation}

\section{Facteurs Affectant la Précision des Calculs}

\subsection{Précision des TLE}

Les données TLE ont une précision limitée et se dégradent avec le temps. En général:
\begin{itemize}
    \item Pour les orbites basses (LEO): Précision acceptable pendant environ 3-5 jours
    \item Pour les orbites géostationnaires: Précision acceptable pendant environ 1-2 semaines
\end{itemize}

\subsection{Modèles Physiques}

La précision des calculs dépend également de la fidélité des modèles physiques utilisés:
\begin{itemize}
    \item Modèle de géopotentiel (jusqu'à quel degré/ordre)
    \item Inclusion ou non des forces de marée
    \item Précision du modèle d'atmosphère pour la traînée
    \item Prise en compte de la pression de radiation solaire
\end{itemize}

\subsection{Paramètres d'Intégration}

Les paramètres de l'intégrateur numérique affectent également la précision:
\begin{itemize}
    \item Tolérance relative et absolue
    \item Taille minimale et maximale du pas d'intégration
    \item Méthode d'interpolation utilisée
\end{itemize}

\section{Exemple de Calcul Complet}

\subsection{Configuration Initiale}

\begin{verbatim}
// Initialisation des données TLE
TLE tle = new TLE("1 25544U 98067A   21086.42859439  .00000318  00000-0  16195-4 0  9991",
                  "2 25544  51.6445 354.6522 0003156  48.6176  25.4760 15.48939243277469");

// Conversion en orbite
TLEPropagator propagator = TLEPropagator.selectExtrapolator(tle);

// Définition de la période d'analyse
AbsoluteDate startDate = new AbsoluteDate(2023, 1, 1, 0, 0, 0.0, TimeScalesFactory.getUTC());
AbsoluteDate endDate = startDate.shiftedBy(86400.0); // +24h

// Création d'un collecteur d'événements d'éclipse
List<AbsoluteDate> eclipseEntries = new ArrayList<>();
List<AbsoluteDate> eclipseExits = new ArrayList<>();
\end{verbatim}

\subsection{Configuration du Détecteur d'Éclipses}

\begin{verbatim}
// Obtention des corps célestes
PVCoordinatesProvider sun = CelestialBodyFactory.getSun();
PVCoordinatesProvider earth = CelestialBodyFactory.getEarth();

// Création du détecteur d'éclipses
double earthRadius = Constants.RIGID_EARTH_SEMI_MAJOR_AXIS;
EclipseDetector eclipseDetector = new EclipseDetector(sun, earthRadius, earth)
    .withMaxCheck(60.0)
    .withThreshold(1.0e-3)
    .withHandler((s, detector, increasing) -> {
        if (increasing) {
            // Sortie d'éclipse
            eclipseExits.add(s.getDate());
        } else {
            // Entrée en éclipse
            eclipseEntries.add(s.getDate());
        }
        return Action.CONTINUE;
    });

// Ajout du détecteur au propagateur
propagator.addEventDetector(eclipseDetector);
\end{verbatim}

\subsection{Propagation et Collecte des Résultats}

\begin{verbatim}
// Propagation de l'orbite
SpacecraftState finalState = propagator.propagate(startDate, endDate);

// Traitement des résultats
List<EclipsePeriod> eclipsePeriods = new ArrayList<>();
for (int i = 0; i < Math.min(eclipseEntries.size(), eclipseExits.size()); i++) {
    AbsoluteDate entryDate = eclipseEntries.get(i);
    AbsoluteDate exitDate = eclipseExits.get(i);
    
    double durationInSeconds = exitDate.durationFrom(entryDate);
    
    EclipsePeriod period = new EclipsePeriod();
    period.setEntryTime(entryDate.toString());
    period.setExitTime(exitDate.toString());
    period.setDurationMinutes(durationInSeconds / 60.0);
    
    eclipsePeriods.add(period);
}
\end{verbatim}

\section{Conclusion}

Le calcul des périodes d'éclipses satellitaires est un problème d'astrodynamique qui combine:

\begin{itemize}
    \item Une modélisation géométrique de l'ombre de la Terre
    \item La propagation précise des orbites satellitaires
    \item Des techniques avancées de détection d'événements
\end{itemize}

Orekit fournit une implementation robuste et précise qui permet de:
\begin{itemize}
    \item Calculer les fenêtres d'éclipse pour la planification des missions
    \item Estimer les durées d'éclipse pour le dimensionnement des systèmes d'alimentation
    \item Analyser l'impact des éclipses sur les opérations satellitaires
\end{itemize}

Ces calculs sont essentiels pour la conception, la planification et l'exploitation des systèmes spatiaux.

\end{document}
